\documentclass[../main.tex]{subfiles}

\begin{document}

\chapter{Divisbility}
Mostly, in every book, they introduce divisbility first. This is because
number theory are based on divisibility e.g. see if $9$ is divisible by $3$
or not. This chapter can be very complicated to understand.

\section{Divisors and Multiples}
Let's introduce new symbol.
\begin{mybox}
	\[ m \mid n \]
	implies m is a divisor of n \textbf{or} n is divisible by m.
\end{mybox}

But what is a \textit{divisor}? To understand this,
let's see an example. Suppose two number $-9$ and $3$. How if $-9 \div 3$?
The answer is $-3$. Thus, we know that:

\begin{center}
	$-9$ is divisible by $3$.
\end{center}

Thus, $3$ is a divisor of $-9$. This means:
\begin{definition}
	\textbf{Divisors(also known as factors)} are numbers that evenly
	divide a larger number.
\end{definition}
You can also express it as $3 \mid -9$. We will see another example. Let's say:
how if $7 \div 2$? It will give out remainder of $1$. Thus, $2$ is not a divisor
of $7$. You can also express it as $2 \nmid 7$.

We can do some problems first.

\begin{problem}
Determine whether:
\begin{multicols}{2}
	\begin{enumerate}[label=(\alph*)]
		\item $6$ is a divisor of $42$
		\item $8$ is a divisor of $-52$
		\item $9$ is a divisor of $36$
		\item $1$ is a divisor of $91$
	\end{enumerate}
\end{multicols}
\end{problem}

\begin{problem}
Determine whether:
\begin{multicols}{2}
	\begin{enumerate}[label=(\alph*)]
		\item $9 \mid 21$
		\item $2 \mid 14$
		\item $3 \nmid -33$
		\item $7 \nmid 63$
	\end{enumerate}
\end{multicols}
\end{problem}



\end{document}
